\section{Introduction}
Automatic language identification is a challenging problem and especially discriminating between closely related languages is one of the main bottlenecks of state-of-the-art language identification systems.\cite{DSL2014}\\

The problem has been investigated in recent work \cite{DSLEvaluation}\cite{DSL2015} which discuss the results from two editions of the "Discriminating between Similar Languages (DSL) shared task". Over the two editions of the DSL shared task different teams competed to develop the best machine learning algorithms to discriminate between the languages in a corpus consisting of 20K sentences in each of the languages: Bosnian, Croatian, Serbian, Indonesian, Malaysian, Czech, Slovak, Brazil Portuguese, European Portuguese, Argentine Spanish, Peninsular Spanish, Bulgarian and Macedonian.\\

In this project we will develop a machine learning based pipeline for automatic language identification for the Nordic languages. Concretely we will focus on discrimination between the six Nordic languages: Danish, Swedish, Norwegian (Nynorsk), Norwegian (Bokmål), Faroese and  Icelandic.\\

We will explore different ways of extracting features from a corpus of raw text data consisting of summaries from Wikipedia in the respective languages and continue to evaluate the performance of a selection of machine learning models.\\

Concretely we will compare the performance of classic machine learning models such as Logistic Regression, Naive Bayes, Support vector machine, and K nearest Neighbors with more contemporary neural network approaches such as Multilayer Perceptrons (MLP)  Convolutional Neural Networks (CNNs).\\

After evaluating these models on the Wikipedia dataset we will continue to evaluate the best models on a Dataset from a different domain in order to investigate how well the models generalize when classifying sentences from a different domain.


%  and a Long Short Term Memory architecture (LSTM).\\
% Feed Forward Neural networks or
