\documentclass[10pt, a4paper]{article}



\usepackage{lrec}
%\usepackage{multibib}
%\newcites{languageresource}{Language Resources}
\usepackage{graphicx}
\usepackage{tabularx}
\usepackage{soul}
% for eps graphics
\usepackage{subcaption}

\usepackage{amsmath}

\usepackage[T1]{fontenc}

\usepackage{epstopdf}
%\usepackage[latin1]{inputenc}
\usepackage[utf8]{inputenc} %% Render certain characters correct.
\usepackage{hyperref}
\usepackage{xstring}

\newcommand{\secref}[1]{\StrSubstitute{\getrefnumber{#1}}{.}{ }}

\title{Discriminating Between Similar Nordic Languages}

\name{René Haas, Leon Derczynski}

\address{IT University of Copenhagen \\
         \{renha, leod \}@itu.dk\\}


\abstract{Automatic language identification is a challenging problem and especially discriminating between closely related languages is one of the main bottlenecks of state-of-the-art language identification systems. This paper presents a machine learning approach for automatic language identification for the Nordic languages, which often suffer miscategorisation by existing state-of-the-art tools. Concretely we will focus on discrimination between the six Nordic languages: Danish, Swedish, Norwegian (Nynorsk), Norwegian (Bokmål), Faroese and Icelandic.
\\ 
\newline \Keywords{DSL, NLP, Language Identification, Machine Learning} }

\begin{document}

\maketitleabstract

\section{Introduction}
Automatic language identification is a challenging problem and especially discriminating between closely related languages is one of the main bottlenecks of state-of-the-art language identification systems.\cite{DSL2014}\\

This paper presents a machine learning approach for automatic language identification for the Nordic languages. Concretely we will focus on discrimination between the six Nordic languages: Danish, Swedish, Norwegian (Nynorsk), Norwegian (Bokmål), Faroese and Icelandic.\\

This papers explore different ways of extracting features from a corpus of raw text data consisting of Wikipedia summaries in respective languages and evaluates the performance of a selection of machine learning models.\\

Concretely we will compare the performance of classic machine learning models such as Logistic Regression, Naive Bayes, Support vector machine, and K nearest Neighbors with more contemporary neural network approaches such as Multilayer Perceptrons (MLP) and Convolutional Neural Networks (CNNs).\\

After evaluating these models on the Wikipedia data set we will continue to evaluate the best models on a data set from a different domain in order to investigate how well the models generalize when classifying sentences from a different domain.

\section{Related Work}

The problem has been investigated in recent work \cite{DSLEvaluation}\cite{DSL2015} which discuss the results from two editions of the ``Discriminating between Similar Languages (DSL) shared task". Over the two editions of the DSL shared task different teams competed to develop the best machine learning algorithms to discriminate between the languages in a corpus consisting of 20K sentences in each of the languages: Bosnian, Croatian, Serbian, Indonesian, Malaysian, Czech, Slovak, Brazil Portuguese, European Portuguese, Argentine Spanish, Peninsular Spanish, Bulgarian and Macedonian.

\section{The data set}
The first step when constructing a language classifier is to gather a data set. For this purpose we use a small script using the Wikipedia API for Python.\footnote{\url{https://pypi.org/project/wikipedia/}}

The script helped download the summaries for randomly chosen Wikipedia articles in each of the languages which are saved to as raw text to 6 {\tt .txt} files of about 10MB each.

After the initial cleaning, described in the next section, the data set contains just over 50K sentences in each of the language categories. From this two data sets with exactly 10K and 50K sentences respectively are drawn from the raw data set. In this way the data sets we will work with are balanced, containing the same number of data points in each language category.

Throughout this report we split these data sets, reserving 80\% for the training set and 20\% for the test set we use when evaluating the models.

\subsection{Data Cleaning}
This section describes how the data set is initially cleaned and how sentences are extracted from the raw data.

\paragraph{Extracting Sentences}

The first thing we want to do is to divide the text into sentences.
This is generally a non-trivial thing to do. Our approach is to first split the raw string by line break.
This roughly divides the text into paragraphs with some noise which we filter out later.

We then extract shorter sentences with the sentence tokenizer ({\tt sent\_tokenize}) function from the NLTK\cite{nltk} python package. This does a better job than just splitting by {\tt '.'} due to the fact that abbreviations, which can appear in a legitimate sentence, typically include a period symbol.

\paragraph{Cleaning characters}
The initial data set have a lot of characters that do not belong to the alphabets of the languages we work with. Often the Wikipedia pages for people or places contain the name in the original language. For example a summary might contain Chinese or Russian characters which are arguably irrelevant for the purpose of discriminating between the Nordic languages.

To make the feature extraction simpler, and to reduce the size of the vocabulary, the raw data is converted to lowercase and stripped of all characters with are not part of the standard alphabet of the six languages.

In this way we only accept the following character set
\begin{verbatim}
'abcdefghijklmnopqrstuvwxyzáäåæéíðóöøúýþ '
\end{verbatim}
and replace everything else with white space before continuing to extract the features.
For example the raw sentence
\begin{verbatim}
'Hesbjerg er dannet ved sammenlægning
 af de 2 gårde Store Hesbjerg
 og Lille Hesbjerg i 1822.'
\end{verbatim}
will after this initial cleanup be reduced to
\begin{verbatim}
'hesbjerg er dannet ved sammenlægning
 af de gårde store hesbjerg
 og lille hesbjerg i ',
\end{verbatim}

We thus make the assumption that capital letters, numbers and characters outside this character set do not contribute much information relevant for language classification.

\section{Baselines}

\subsection{Baseline With langid.py}

As a baseline to compare the performance of the models in we compare with an out of the box language classification system. ``langid.py: An Off-the-shelf Language Identification Tool." \cite{langID} is such a tool.\\

Out of the box langid.py comes with with a pretrained model which covers 97 languages. The data for langid.py comes from from 5 different domains: government documents, software documentation, newswire, online encyclopedia and an internet crawl.\\

We evaluated how well langid.py performed on the Wikipedia dataset. Since langid.py returned the language id ``no" (Norwegian) on some of the data points we restrict langid.py to only be able to return either ``nn" (Nynorsk) or ``nb" (Bokmål) as predictions. It is a quite peculiar feature of the Norwegian language that there exist two different written languages but three different language codes.\\

\begin{figure}[h!]
  \centering
  \includegraphics[width = 200pt]{figs/langid}
  \caption{Confusion matrix with results from langid.py on the full dataset with 300K data points. }
  \label{langid_confusion_matrix}
\end{figure}

In Figure~\ref{langid_confusion_matrix} we see the confusion matrix for the langid.py classifier. The largest errors are between Danish and Bokmål and between Faroese and Icelandic. We see that langid.py was actually able to correctly classify most of the Danish data points however approximately a quarter of the data points in Bokmål was incorrectly classified as Danish and just under and eighth was classified as Nynorsk.\\

Furthermore langid.py correctly classified most of the Icelandic data points however over half of the data points in Faroese was incorrectly classified as Icelandic.\\

\begin{table}
  \centering
  \begin{tabular}{ l | c | r }
    \hline
    Model               & Encoding  & Accuracy \\
    \hline
    Knn                 & cbow &  0.780\\
    Log-Reg             & cbow &  0.819\\
    Naive Bayes         & cbow &  0.660\\
    SVM                 & cbow &  0.843\\
    Knn                 & skipgram &  0.918\\
    Log-Reg             & skipgram &  \textbf{0.929}\\
    Naive Bayes         & skipgram &  0.840\\
    SVM                 & skipgram &  \textbf{0.928}\\
    Knn                 & char bi-gram  & 0.745\\
    Log-Reg             & char bi-gram  & 0.907\\
    Naive Bayes         & char bi-gram  & 0.653\\
    SVM                 & char bi-gram  & 0.905\\
    Knn                 & char uni-gram  & 0.620\\
    Log-Reg             & char uni-gram  & 0.755\\
    Naive Bayes         & char uni-gram  & 0.614\\
    SVM                 & char uni-gram  & 0.707\\
    \hline
  \end{tabular}
  \caption{Overview of results for the dataset with 10K data points in each language.}
  \label{baseline-results-10k}
\end{table}

\subsection{Baseline with linear models}

In the Table~\ref{baseline-results-10k} we see the results for running the models on a dataset with 10K data points in each language category. We see that the models tend to perform better if we use character bi-grams instead of single characters.\\

Also we see that logistic regression and support vector machines outperform Naive Bayes and K-nearest neighbors in all cases. Furthermore for all models we get the best performance if we use the skip-gram model from FastText.\\

Comparing the CBOW mode from FastText with character bi-grams we see that the CBOW model is on par with bi-grams for the KNN and Naive Bayes classifiers while bi-grams outperform  CBOW for Logistic Regression and support vector machines.\\


\section{Our Approach}

\subsection{Using FastText}

The methods described above are quite simple. We also compare the above method with FastText, which is a library for creating word embeddings developed by Facebook~\cite{BagOfTricks}. 

In the paper "Enriching Word Vectors with Subword Information" \cite{EnrichingWordVectors} the authors explain how FastText extracts feature vectors from raw text data. FastText makes word embedding using one of two model architectures: continuous bag of words (CBOW) or the continuous skip-gram model.

The skip-gram and CBOW models are first proposed in \cite{EfficientWordRepresentations} which is the paper introducing the word2vec model for word embeddings. FastText builds upon this work by proposing an extension to the skip-gram model which takes into account sub-word information.

Both models use a neural network to learn word embedding from using a context windows consisting of the words surrounding the current target word. The CBOW architecture predicts the current word based on the context, and the skip-gram predicts surrounding words given the current word.\cite{EfficientWordRepresentations}

\subsection{Using A Convolutional Neural Network}

While every layer in the MLP is densely connected such that each of the nodes in a layer is connected to all nodes in the next layer, in a convolution neural network we use one or more convolutional layers.

Convolutional Neural networks are very popular for image recognition but they can also be used for text classification \cite{textcnn_google}.

\begin{figure}[h!]
  \centering
  \includegraphics[width = 200pt]{figs/cnn_diagram}
  \caption{Diagram of Convolutional Neural network.}
  \label{cnn}
\end{figure}

The basic premise of a convolutional layer is illustrated in Figure~\ref{cnn}.\footnote{Source: \url{https://realpython.com/python-keras-text-classification/}} In a CNN you have a filter which slides over the input. The CNN then takes the dot product of the weights of the filter and the corresponding input features, before applying the activation function.

\section{Results}
In this section I will present the results from running the various classification models on the Wikipedia dataset and then proceed to investigate how well these models perform on a dataset from a different domain.

\subsection{Baseline With langid.py}

As a baseline to compare the performance of the models in this word I would like to compare with an out of the box language classification system. "langid.py: An Off-the-shelf Language Identification Tool." \cite{langID} is such a tool.\\

Out of the box langid.py comes with with a pretrained model which covers 97 languages. The data for langid.py comes from from 5 different domains: government documents, software documentation, newswire, online encyclopedia and an internet crawl.\\

I have written a program to test how well langid.py performed on the Wikipedia dataset using the the python API.\footnote{\url{https://pypi.org/project/langid/}}\\

Since langid.py returned the language id "no" (Norwegian) on some of the data points i chose to restrict langid.py to only be able to return either "nn" (Nynorsk) or "nb" (Bokmål) as predictions. It is a quite peculiar feature of the Norwegian language that there exist two different written languages but three different language codes.

\begin{figure}[h!]
  \centering
  \includegraphics[width = 200pt]{figs/langid}
  \caption{Confusion matrix with results from langid.py on the full dataset with 300K data points. }
  \label{langid_confusion_matrix}
\end{figure}

In Figure \ref{langid_confusion_matrix} we see the confusion matrix for the langid.py classifier. The largest errors are between Danish and Bokmål and between Faroese and Icelandic. We see that langid.py was actually able to correctly classify most of the Danish data points however approximately a quarter of the data points in Bokmål was incorrectly classified as Danish and just under and eighth was classified as Nynorsk.\\

Furthermore langid.py correctly classified most of the Icelandic data points however over half of the data points in Faroese was incorrectly classified as Icelandic.

\subsection{Baseline with linear models.}

In the Figure \ref{baseline-results-10k} we see the results for running the models on a dataset with 10K data points in each language category. We see that the models in general perform better if we use character level bi-grams instead of uni-grams (individual) characters.\\

Also we see that logistic regression and support vector machines outperform Naive Bayes and K nearest neighbors in all cases. Furthermore for all models we get the best performance if we use the skipgram model from FastText.\\

If we compare the cbow mode from FastText with character level bi-grams we see that the cbow model is on par with bi-grams for the KNN and Naive Bayes classifiers while bi-grams outperform the cbow for Logistic Regression and support vector machines.\\

\begin{figure}[h!]
  \centering
  \begin{tabular}{ l | c | r }
    \hline
    Model               & Encoding  & Accuracy \\
    \hline
    Knn                 & cbow &  0.780\\
    Log-Reg             & cbow &  0.819\\
    Naive Bayes         & cbow &  0.660\\
    SVM                 & cbow &  0.843\\
    Knn                 & skipgram &  0.918\\
    Log-Reg             & skipgram &  \textbf{0.929}\\
    Naive Bayes         & skipgram &  0.840\\
    SVM                 & skipgram &  \textbf{0.928}\\
    Knn                 & char bi-gram  & 0.745\\
    Log-Reg             & char bi-gram  & 0.907\\
    Naive Bayes         & char bi-gram  & 0.653\\
    SVM                 & char bi-gram  & 0.905\\
    Knn                 & char uni-gram  & 0.620\\
    Log-Reg             & char uni-gram  & 0.755\\
    Naive Bayes         & char uni-gram  & 0.614\\
    SVM                 & char uni-gram  & 0.707\\
    \hline
  \end{tabular}
  \caption{Overview of results for the dataset with 10K data points in each language.}
  \label{baseline-results-10k}
\end{figure}

\subsection{Results with neural networks.}
In the following we evaluate the initial results for the neural network architectures which can be seen in Figure \ref{keras-results}. Here we compare the result of doing character level uni- and bi-grams using the Multilayer Perceptron and Convolutional Neural Network. We see that the CNN performs the best, achieving an accuracy of 95.6\% when using character bi-grams. Both models perform better using bi-grams than individual characters as features while the relative increase in performance is greater for the MLP model.

\begin{figure}[h!]
  \centering
  \begin{tabular}{ l | c | r }
    \hline
    Model               & Encoding  & Accuracy \\
    \hline
    MLP                 & char bi-gram &  0.898 \\
    CNN                 & char bi-gram & \textbf{0.956} \\
    % LSTM                & char bi-gram & 60k & 0.933 \\
    MLP                 & char uni-gram &  0.697\\
    CNN                 & char uni-gram  & 0.942 \\
    % LSTM                & char uni-gram & 60k &  \\
    \hline
  \end{tabular}
  \caption{Overview of results for the neural network models for the dataset with 10K data points in each language.}
  \label{keras-results}
\end{figure}


\subsection{Increasing the size of the dataset.}
Usually the performance of supervised classification models increase with more training data. In this section I have increased amount of training data to 50K sentences in each of the language categories. Due to much longer training times I choose to only rerun the "baseline models" with the skipgram encoding from FastTest which we saw achieved the highest accuracy.

\begin{figure}[h!]
  \centering
  \begin{tabular}{ l  c | r }
    \hline
    Model               & Encoding & Accuracy \\
    \hline
    Knn                 & skipgram & 0.931\\
    Logistic Regression & skipgram  & \textbf{0.933}\\
    Naive Bayes         & skipgram  & 0.806\\
    SVM                 & skipgram& 0.925\\
    \hline
  \end{tabular}
  \caption{Overview of results for the dataset with 50K data points in each language.}
  \label{results-sklearn300k}
\end{figure}

As can be seen in Figure \ref{results-sklearn300k} the accuracies for the logistic regression model and the K nearest neighbors algorithm improved by including more data, however not by much. Unexpectedly the accuracies for the support vector machine and naive Bayes actually dropped a bit by including more data.\\

Even when including five times the amount of data the best result, logistic regression with an accuracy of 93.3\%, is still worse than for the Convolutional Neural Network trained on 10K data points in each language.\\

In Figure \ref{results-keras-300k} we see the results for running the neural networks on the larger dataset. Both models improve by increasing the amount of data and the Convolutional Neural Network reached an accuracy of 97\% which is the best so far.

\begin{figure}[h!]
  \centering
  \begin{tabular}{ l c | r }
    \hline
    Model               & Encoding & Accuracy \\
    \hline
    MLP                 & char bi-gram  & 0.918\\
    CNN                 & char bi-gram  & \textbf{0.970}\\
    \hline
  \end{tabular}
  \caption{Overview of results for the dataset with 50K data points in each language.}
  \label{results-keras-300k}
\end{figure}

% In Figure \ref{training} we see a plot of the accuracy during training on the training and validation set respectively. We see that the accuracy on the validation set follows the accuracy on the training set more closely for the CNN than for the MLP.

% \begin{figure}[h!]
%     \centering
%     \begin{subfigure}[b]{100pt}
%         \includegraphics[width=\textwidth]{figs/training_MLP}
%         \caption{Training accuracy MLP}
%     \end{subfigure}
%     ~
%     \begin{subfigure}[b]{100pt}
%         \includegraphics[width=\textwidth]{figs/training_CNN}
%         \caption{Training accuracy CNN}
%     \end{subfigure}
%     \caption{Accuracy on test and validation set during training of the different neural networks on the full dataset with 50K data points in each language.}
%     \label{training}
% \end{figure}

In Figure \ref{confusion_matrix-big-cnn} we see the confusion matrix for the convolutional Neural Network trained on the full Wikipedia dataset with 300K data points pr language. We see that the largest classification errors still happens between Danish, Bokmål and Nynork as well as between Icelandic and Faroese. \\

\begin{figure}[h!]
  \centering
  \includegraphics[width = 200pt]{figs/confusion_CNN}
  \caption{Confusion matrix with results from the convolutional neural network on the full datasetwith 50K data points in each language.}
  \label{confusion_matrix-big-cnn}
\end{figure}

\subsection{Optimizing the kernel size.}
In attempt to optimize the CNN i tried to look at different kernel sizes (between 1 and 11) for character level uni- bi- and trigrams. I tested on the dataset with 10K sentences in each language since the training time for a single CNN was several hours and it would have taken several days to train on the full dataset even using cloud resources in Google Colab. We see the results in Figure \ref{kernel_sizes}.\\

\begin{figure}[h!]
  \centering
  \includegraphics[width = 225pt]{figs/KernelSizes}
  \caption{Accuracy for different kernel sizes for the Constitutional Neural Network.}
  \label{kernel_sizes}
\end{figure}

\subsection{Using FastText supervised}

FastText can also be used for be trained supervised and be used for Classification. In the Paper Bag of Tricks for Efficient Text Classification \cite{BagOfTricks} the authors show that FastText can obtain performance on par with methods inspired by deep learning, while being much faster on a selection of different tasks in Tag prediction and Sentiment Analysis. The confusion matrix from running the FastText supervised classifier can be seen in Figure \ref{fasttext_supervised}. We see that FastText is on par with the CNN\\


\begin{figure}[h!]
  \centering
  \includegraphics[width = 225pt]{figs/fasttext_supervised}
  \caption{Confusion matrix with results from a supervised FastText model on the full dataset with 300K datapoints.}
  \label{fasttext_supervised}
\end{figure}

\subsection{Reflections.}
At this point I it beneficial to take a step back and look at what we have done so far. We have developed a selection of different models and our results to far indicate that the FastText supervised model and the CNN both give quite good results on the Wikipedia dataset with accuracies over 97\%. \\

In the rest of this section I would like to investigate the cases there these classifiers fail and try if I can find common patters. Also i will test the models on a dataset which they have not been trained on to investigate how well the models generalize to data from a different domain than Wikipedia. \\


\subsection{Looking at failure cases}
Now having achieved quite good results with FastText and the CNN I now look a bit more closely at the sentences which were misclassified by both models. \\

As a first observation the misclassified sentences tend to be shorter than the correctly classified ones. In Figure \ref{faliurelengthdist} I have plotted the distribution of the length, defined as the number of characters after the data cleaning procedure as defined in a previous section. The mean length of the sentences in the test set is 104.74 characters with a standard deviation of 65.39 while it is only 48.91 characters with a deviation  of 37.27. \\

\begin{figure}[h!]
    \centering
    \includegraphics[width=200pt]{figs/faliurelengthdist}
    \caption{Distribution of sentence lengths.}
    \label{faliurelengthdist}
\end{figure}

Generally the misclassified sentences fall into the categories: Very short sentences, names, mathematical content and sentences with many foreign words.\\

\textbf{Mathematical equations}:
A substantial (and useful) part of Wikipedia is related to mathematics and equations are common on these pages. My Wikipedia scraper did not filter out mathematical equations which appear in the dataset as in the examples below. \\

\begin{verbatim}
displaystyle n lambda eff cdot t
displaystyle a frac a cot frac pi simeq a
displaystyle a b c a cdot b a cdot c ab ac
\end{verbatim}

\textbf{Foreign Words}:
Another common type of misclassified sentences with a lot of foreign words. In the dataset this commonly happened with English words. Below are some examples of misclassified sentences where most or all word are from foreign languages. \\

\begin{verbatim}
art garfunkel all i know skywriter
 watermark

mémoires de la société des antiquaires
du nord fjögur bindi

avatar the last airbender prins zuko

the fifth dimension up up and away
\end{verbatim}
% heimsnet e mbwa mobile broadband wireless access
% y is yhat dk
%  y is yhat nb
% y dk yhat fo
%  y nb yhat fo


\textbf{Names}: Another common failure case is sentences that contain only names of people or locations. Examples are
\begin{verbatim}
  romain rolland
  henry morton stanley
  anna karin
\end{verbatim}
% y is yhat fo
% y sv yhat fo
% y is yhat fo

It is not surprising that these cases are hard to classify since names are usually spelled the same way irrespective of language. \\

\textbf{Danish and Bokmål}:
One of the largest error categories was sentences on Danish or Bokmål misclassified as the other language.\\

This is not surprising sine these two languages are very closely related and often the difference between them is an alternative spelling of a single word in a sentence. \\

The examples below are in Bokmål but was classified as Danish. The first sentence is in indistinguishable from danish except from the alternative spelling of the single word  "magesekken" which would be "mavesækken" in Danish. \\

The Second sentence would be hard or impossible for a native speaker to distinguish since the there is no difference between this sentence in the two languages.Finally for the third example the difference is only two characters in the word "sommerfuglhager" which in danish would be spelled as "sommerfuglehaver".

\begin{verbatim}
(1) hos mennesket har magesekken et
    volumet ca

(2) klubben har hjemmebane på slettebakken
    hvor de også har et klubbhus

(3) en dyrepark kan derfor også omfatte
    for eksempel akvarier terrarier
    sommerfuglhager og fuglehus
\end{verbatim}

\subsection{Evaluating the models on another dataset}
It would be interesting to see how the two best performing models generalize by testing on a dataset different from the they have been trained on (the Wikipedia dataset).\\

I downloaded an additional dataset from Tatoeba\footnote{{\tt tatoeba.org/}} which is a large database of user provided sentences and translations. In Figure \ref{tatoebasentprlang} we see the number of sentences in each language for all sentences in the Tatoeba dataset. Observe that we have very few samples in Nynorsk and Faroese.\\

\begin{figure}[h!]
    \centering
    \begin{subfigure}[b]{0.45\textwidth}
      \includegraphics[width =\textwidth]{figs/tatoebasentprlang}
      \caption{Distribution of the number of sentences in each language in the tatoeba dataset.}
      \label{tatoebasentprlang}
    \end{subfigure}
    ~
    \begin{subfigure}[b]{0.45\textwidth}
      \includegraphics[width =\textwidth]{figs/taboeba-faliurelengthdist}
      \caption{Distribution of the length of  sentences tatoeba dataset.}
      \label{tatoebalengths}
    \end{subfigure}
        \caption{Distribution of the lengths and language classes of Tatoeba sentences.}
\end{figure}

The language used in the Tatoeba dataset is different from the language used in Wikipedia. The Tatoeba dataset mainly consists of sentences written in everyday language. Below we see some examples from the Danish part of the Tatoeba dataset.
\begin{verbatim}
Hvordan har du det?

På trods af al sin rigdom
og berømmelse, er han ulykkelig.

Vi fløj over Atlanterhavet.

Jeg kan ikke lide æg.

Folk som ikke synes at latin er det
smukkeste sprog, har intet forstået.
\end{verbatim}

The confusion matrix for the FastText supervised model and the CNN, which are both trained on the full Wikipedia dataset and then evaluated on the Tatoeba dataset can be seen in Figure \ref{tatoeba-confuss}. Both models use the same setting that produced good results on the Wikipedia dataset.

\begin{figure}[h!]
    \centering
    \begin{subfigure}[b]{0.45\textwidth}
        \includegraphics[width=\textwidth]{figs/tatoeba-langid}
        \caption{Langid.py}
    \end{subfigure}
    ~
    \begin{subfigure}[b]{0.45\textwidth}
        \includegraphics[width=\textwidth]{figs/tatoeba-fasttext}
        \caption{fasttext classifier }
    \end{subfigure}
    ~
    \begin{subfigure}[b]{0.45\textwidth}
        \includegraphics[width=\textwidth]{figs/tatoeba-cnn}
        \caption{Convolutional neural network}
    \end{subfigure}
    \caption{Confusion matrices for the tatoeba dataset }
    \label{tatoeba-confuss}
\end{figure}

We see that the performance drops quite a lot when shifting to another domain. For reference the accuracy of langid.py on this dataset is 80.9\% so FastText actually performs worse than the baseline with an accuracy of 75.5\% while the CNN is a bit better than the baseline with an accuracy of 83.8 \% \\

One explanation for the drop in performance is that the sentences in the Tatoeba dataset is significantly shorter than the sentences in the Wikipedia dataset as seen in Figure \ref{tatoebalengths}. As we saw in the previous section both models tend to misclassify shorter sentences more often than longer sentences. This and the fact that the "genre" of sentences are different might explain why the models trained on the Wikipedia dataset does not generalize too well to the Tatoeba dataset without a drop on performance.\\

The CNN uses character bi-grams as features while, with the standard settings, FastText uses only individual words to train. The better performance of the CNN might indicate that character level n-grams are more useful features for language identification than words alone.\\

To test this I changed the setting of FastText to train using only character level n-grams in the range 1-5 instead of individual words. In Figure \ref{fasttextcharngram} we see the confusion matrix for this version of the FastText model. This version still achieved 97.8\% on the Wikipedia test set while improving the accuracy on the Tatoeba dataset from 75.4\% to 85.8\% which is a substantial increase.\\

Thus using character level features seem to improve the FastText models ability to generalize to sentences belonging to a domain different from the one it has been trained on.

\begin{figure}[h!]
    \centering
    \includegraphics[width=200pt]{figs/fasttextcharngram}
    \caption{Confusion matrix for FastText trained using only characterlevel n-grams on the Wikipedia dataset and evaluated on the tatoeba dataset.}
    \label{fasttextcharngram}
\end{figure}


\subsection{Retraining on the combined dataset}
To improve the accuracy on the Tatoeba dataset I decided to retrain the FastText model on a combined dataset consisting of datapoints from
both the Wikipedia and Tatoeba dataset.\\

The FastText model achieved an accuracy of 97.2\% on this combined dataset and an accuracy of 93.2\% when evaluating this model on the Tatoeba test set alone - the confusion matrix can be seen in Figure \ref{retrain-confuss}.\\

As was the case with the Wikipedia dataset the misclassified sentences tend to be shorter than the average sentence in the dataset. In Figure \ref{retrain-lengths} we see the distribution of sentence lengths for the Tatoeba test set along with the misclassified sentences.\\

 In the Tatoeba test set the mean length of sentences is 37.66 characters with a standard deviation of 17.91 while the mean length is only 29.70 characters for the misclassified sentences with a standard deviation of 9.65. This again supports the conclusion that shorter sentences are harder to classify.

\begin{figure}[h!]
    \begin{subfigure}[b]{0.45\textwidth}
      \includegraphics[width=200pt]{figs/retrain}
      \caption{Confusion matrix for FastText trained using only characterlevel n-grams on the combined Wikipedia/Tatoeba dataset and evaluated on the tatoeba dataset.}
      \label{retrain-confuss}
    \end{subfigure}
    ~
    \begin{subfigure}[b]{0.45\textwidth}
        \includegraphics[width=\textwidth]{figs/faliurelengthdist_tatoeba}
        \caption{Distribution of sentence lengths Tatoeba testset along with the misclassified sentences.}
     \label{retrain-lengths}
    \end{subfigure}
\end{figure}


% Not surprisingly we see a substantial increase in performance when including datapoints

% \begin{figure}[h!]
%     \centering
%     \includegraphics[width=200pt]{figs/retrain}
%     \caption{Confussion matrix for fastText trained using only characterlevel ngrams on the combined Wikipedia/Tatoeba dataset and evaluated on the tatoeba dataset.}
%     \label{retrain}
% \end{figure}



% \begin{figure}
%   \centering
%   \includegraphics[width = 300pt]{figs/encoding_dimension}
%   \caption{The the performance of the different models.}
%   \label{encodings}
% \end{figure}
%
%
% \begin{figure}
%   \centering
%   \includegraphics[width = 300pt]{figs/skipgram}
%   \caption{The the performance of the different models.}
%   \label{skipgram}
% \end{figure}


\section{Analysis}

\subsection{Principal Component analysis and t-SNE}
To gain additional insight on how the different word embedding capture important information about each of the language classes we thought it would be interesting to try and visualize the embeddings using two different techniques for dimensionality reduction.

No matter which way we choose to extract the feature vectors they belong to a high dimensional feature space and in order to do visualization we need to project the feature vectors down to 2d space.

To do this we have implemented two different methods: Principal Component Analysis (PCA) which we will compare with T-distributed Stochastic Neighbor Embedding (t-SNE).
Here we will begin with a brief explanation of the two techniques and proceed with an analysis of the results.

\paragraph{Principal Component Analysis}

The first step is to calculate the covariance matrix of the data set.
The components of the covariance matrix is given by

\begin{align}
K_{X_i,X_j} = E[(X_i - \mu_i )(X_j -  \mu_j)]
\end{align}

where $X_{i}$ is the $i$th component of the feature vector and $\mu_{i}$ is the mean of that component.

In matrix form we can thus write the covariance matrix as
\begin{align}
K(\mathbf{x},\mathbf{z}) =
\begin{bmatrix}
    \text{cov}(x_1,z_1) &  \dots  & \text{cov}(x_1,z_n) \\
    \vdots & \ddots     & \vdots \\
    \text{cov}(x_n,z_1) & \dots  & \text{cov}(x_n,z_n) \\
\end{bmatrix}
\end{align}
The next step is to calculate the eigenvectors and eigenvalues of the covariance matrix by solving the eigenvalue equation.
\begin{align}
\det (K v-\lambda v) = 0
\end{align}
The eigenvalues are the variances along the direction of the eigenvectors or "Principal Components". To project our data set onto 2D space we select the two eigenvectors largest associated eigenvalue and project our data set onto this subspace.

In Figure \ref{pca} we see the result of running the PCA algorithm on the wikipedia data set where we have used character level bigrams as features as well as the cbow and skipgram models from FastTest.


\begin{figure}[h!]
    \centering
    \begin{subfigure}[b]{0.47\textwidth}
        \includegraphics[width=\textwidth]{figs/pcachar2}
        \caption{Character bigram}
    \end{subfigure}
    ~
    \begin{subfigure}[b]{0.47\textwidth}
        \includegraphics[width=\textwidth]{figs/pcacbow1}
        \caption{Fasttext cbow}
    \end{subfigure}
    ~
    \begin{subfigure}[b]{0.47\textwidth}
        \includegraphics[width=\textwidth]{figs/pcaskipgram1}
        \caption{Fasttext skipgram}
    \end{subfigure}
    \caption{Dimensionality reduction using PCA}
    \label{pca}
\end{figure}

In the figure for encoding with character level bi-grams the PCA algorithm resulted in two elongated clusters. Without giving any prior information about the language of each sentences the PCA is apparently able to discriminate between Danish, Swedish, Nynorsk and Bokmål on one side and Faroese and Icelandic on the other since the majority of the sentences in each language belong to either of these two clusters. With the FastText implementations we observe three clusters.s

For both cbow and skipgram we see a distinct cluster of Sweedish sentences. When comparing the two FastText models we see that the t-SNE algorithm with skipgrams seems to be able to separate the Faroese and Icelandic data points to a high d ecree compared with the cbow model. Also for the cluster identified with the sentences with Danish, Bokmål and Nynorsk the skipgram models seem seem to give a better separation, however to a lesser degree than with the two former languages.

\paragraph{t-SNE}

The T-distributed Stochastic Neighbor Embedding method was first proposed in 2008 in the paper "Visualizing Data using t-SNE"\cite{tsne}.

In the paper the authors explain the theory behind the algorithm which we  will make a brief summary of here.

Suppose you pick a data point $x_i$, then the probability of picking another data point $x_j$ as a neighbor to $x_i$ is given by
\begin{align}
p_{ji}= \frac{\exp (|| x_i - x_j ||^2/2\sigma_i^2 )}{\sum_{k\neq i} \exp (|| x_i - x_k ||^2/2\sigma_i^2 )}
\end{align}

Now having this probability distribution the goal is to find the low-dimensional mapping of the data points $x_i$ which we denote $y_i$ follow a similar distribution. To solve what is referred to as the "crowding problem" the t-SNE algorithm uses the Student t-distribution which is given by
\begin{align}
q_{ij}= \frac{ (1+|| y_i - y_j ||^2 )^{-1}}{\sum_{k\neq l} (1+|| y_k - y_l ||^2 )^{-1}}
\end{align}
Now finally for optimizing this distribution is done by using gradient decent on the Kullback-Leibler divergence which is given by
\begin{align}
\frac{\delta C}{\delta y_i}= 4 \sum_j (p_{ij} - q_{ij})(y_i-y_j)(1+ || y_i - y_j ||^2  )^{-1}
\end{align}
The result from running the t-SNE algorithm on the Wikipedia data set can be seen in Figure \ref{tsne}. As was the case with the PCA algorithm it appears that the encoding with FastText seem to capture the most relevant information to discriminate between the languages, especially the skip-gram mode seems to do a good job in capturing relevant information.

\begin{figure}[h!]
    \centering
    \begin{subfigure}[b]{0.47\textwidth}
        \includegraphics[width=\textwidth]{figs/tsnechar2}
        \caption{Character bi-gram}
    \end{subfigure}
    ~
    \begin{subfigure}[b]{0.47\textwidth}
        \includegraphics[width=\textwidth]{figs/tsnecbow1}
        \caption{FastText CBOW}
    \end{subfigure}
    ~
    \begin{subfigure}[b]{0.47\textwidth}
        \includegraphics[width=\textwidth]{figs/tsneskipgram1}
        \caption{FastText skip-gram}
    \end{subfigure}
    \caption{Dimensionality reduction using t-SNE}
    \label{tsne}
\end{figure}


Here we recover some interesting information about the similarity of the languages. The data points in Bokmål lies in between those in Danish and Nynorsk while Icelandic and Faroese have their own two clusters which are separated from the three former languages. 

This is in good agreement with what we already know about the languages. Interestingly the Swedish data points and quite scattered and the t-SNE is not able to make a coherent Swedish cluster.

This does not however mean that the Swedish datapoint are not close in the original space. Some care is needed when interpreting the plot since t-SNE groups together data points such that neighboring points in the input space will tend to be neighbors in the low dimensional space.

If points are separated in input space, t-SNE would like to separate them in the low dimensional space however it does not care how far they are separated. So clusters that are far away in the low dimensional space are not necessarily far away in the input space.

\subsection{Discussion}
We used the dimensionality reduction techniques PCA and t-SNE to make visualizations of feature vectors obtained by making a one-hot encoding with character bi-grams and with the two modes from FastText.

These unsupervised techniques was able to separate the sentences from Wikipedia into different clusters.
Without any prior knowledge about the actual language of each sentence these techniques indicated that the six languages can be divided into three main language categories: (1) Danish Nynorsk Bokmål (2) Faroese Icelandic and (3) Swedish.


Generally the supervised models had the largest errors when discriminating between languages belonging to either of the language groups mentioned above.

For the "classical" models we saw that Logistic Regression and support vector machines achieved better performance than Knn and Naive Bayes, where the latter performed the worst. This was true in all cases irrespective of the method of feature extraction.

Additionally we saw that when we used feature vectors from the FastText skip-gram model the classification models achieved better results than when using either FastText CBOW or character n-grams.

Generally we saw that increasing the number of data points lead to better performance. When comparing the CNN with the "classical" models however the CNN performed better than any of the other models even when trained on less data points. In this way it seems that the CNN is able to learn more from less data when compared to the other models.

\section{Conclusions}
\label{sec:conclusion}

%+ tensor model in auxiliary latent space
%+ embedding of BU3DFE ()<maybe mention the compotational cost and resnet to decrease that embedding speed approx  (7500 data points) (2000 epochs)  (8 it/s ) =520hrs /21 days on tesla P100 (colab)

%+ previous findings confirmed: apathy mode 
%+ semantic subspaces
%- ?gradient descent and 
% Einstein Notation? could be its own point since we change the formalism non trivially when we introduce the combined rank 1 parameter tensor
%+ applications: approximation, expression and rotation transfer.

%Summary 
In this work, we have proposed $\tau$GAN a tensor based model for the auxiliary latent space of the StyleGAN. 
We first embedded the face database BU3DFE into the latent space. The resulting latent codes were stored into a multi dimensional  array or the data tensor, whose factorisation by HOSVD then yields semantically meaningful subspaces. We demonstrate the applicability approach by producing intuitive parametrisation for semantic face editing. 
%
%subspace (apathy)
Specifically, we were able to generalise the earlier result \cite{apathy} from feature-based face analysis:  %that suggested an alternative origin for the expressions
the expression subspace of the latent space namely has the structure where the expression trajectories meet in 
a specific emotionless face, the \emph{apathetic} expression, which is different from the mean or neutral face and the face. 
%
We evaluated our approach by quantitatively and qualitatively, and compared different versions of the proposed tensor models on the basis of approximation of unseen samples, and demonstrated stability in transfer of expression and rotation. From the results, we conclude that the proposed approach is powerful way for characterising and parameterising the latent spaces of StyleGAN.
%
%%%%%%%% Future work: keep it super short
In future, we intend to release a database by drawing samples from our auxiliary latent space. 

%\paragraph{Summary}

%\paragraph{Future work}

% \paragraph{Bigger and Better ResNet}
% Embedding a full data set of the scale of BU3D-FE into latent space is a very computationally intensive task. The estimated embedding time with 8it/s and for configuration B is 390 GPU hours on a single Tesla P100. The training of a larger (and hopefully better) ResNet would speed up the process by estimating better initial conditions for the VVG optimization. This would lower "the barrier of entry" for embedding data sets into the StyleGAN latent space.

% \paragraph{Time series data}
% Now, assuming we are able to estimate model parameters for an arbitrary reference image. The next logical step would be to estimate parameters for a time series and see how the parameters evolve. Each face in the tensor model is essentially a linear combination of the subspace basis vectors. We could model the time dependency of the expansion coefficients with Fourier analysis.

% \paragraph{Masking as post processing}
% In the configuration $\alpha$ run of the BU3D-FE latent space embedding we used the masking option from the code by \cite{pbaylies}. Having the unmasked latent vectors, it should be possible to add the mask as a post embedding step. This could be a way to augment the data. 

% \paragraph{Missing data}
% It is possible to define a generalized HOSVD which allows for missing data. This has been investigated in \cite{face2005}. It would be very interesting to implement a generalized model which allows for missing data, as this would make it possible to construct a tensor model on a less structured data set than the BU3D-FE database.  

% \paragraph{Creation of a synthetic data set based on the tensor model}
% We cannot release the tensor model since it is an approximation to the BU3D-FE data set which is under  Copyright by the authors. Our model can however generate synthetic faces with known parameters which do not exist in the BU3D-FE database. In principle we could generate a new synthetic data set, centered around the StyleGAN null\footnote{Mean face of StyleGAN, which ever version.} face, and then construct another tensor model on the synthetic data. 

% % \paragraph{Direct model embedding}
% % Another idea is to extimate the  model parameters directly from image use the VGG approach  to embed directly into our mode.skip the latent code. 


% \paragraph{Variational Auto encoders}
% GANs are not the only promising candidate for an neural network architecture for photo realistic image synthesis. It has been shown that "Vector Quantized" \cite{vqvae} \cite{vqvae2} Variational auto encoder are also able to synthesize high quality images of faces. It would be interesting to test the tensor model approach the latent space of these novel Variational Autoencoders.
 


% \begin{itemize}
%     \item We only consider using the same regularisation parameters along all modes. Maybe results would improve by allowing regularization parameters for each mode
% \item Better grip on regularization parameters: Allow for separate numerical  values for each mode.\footnote{And in case of the extended model- each style. }
% \item Time dependency: We can model the time dependency of model parameters by analyzing video.
% \item The VGG shortcut: We should be able to skip the intermediate embedding into $\mathcal{W}$ space by approximating model parameters directly using the VGG approach 
% \item ResNet: We can train an even larger ResNet for StyleGAN2
% \item Try this model on the StyleGAN $\mathcal{Z}$ space.
% \end{itemize}



\section{References}
\label{main:ref}

\bibliographystyle{lrec}
\bibliography{references}


\end{document}
